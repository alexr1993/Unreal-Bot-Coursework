\documentclass[a4paper,12pt]{article}

\usepackage{natbib}
\usepackage{times}
\usepackage{graphicx,epsfig}
\usepackage[leftcaption]{sidecap}
\usepackage{subfigure} % figures can have sub chunks
\usepackage{geometry} % this maxes page usage, making the below unnecessary
\usepackage{hhline}

\textwidth = 6.75in
\oddsidemargin = -0.25in
\textheight = 10in
\topmargin = -0.5in
\usepackage{fancyhdr}
\pagestyle{fancy}
\lhead{{\it Alex Remedios }}
\chead{Social Behaviour in Games}
\rhead{Coursework 3}
\lfoot{}
\cfoot{\thepage}
\rfoot{}


\newcommand{\goodgap}{%
 \hspace{\subfigtopskip}%
 \hspace{\subfigbottomskip}}
% For FSM diagram
\usepackage{tikz}
\usepackage{bm}

\title{Social Behaviour in Games}
\author{Alex Remedios (ajr36)}
\date{\today}

\begin{document}

% Set this to the language you want to use in your code listings (if any)
% \lstset{language=Java,breaklines,breakatwhitespace,basicstyle=\small}
\maketitle

\section{Introduction}
% Introduce problem


\section{Approach}

% Avoid discussing results here, keep the methodology at the forefront

\section{Results}

% Don't speculate, factual descriptions only

% Describe behaviour with effort

% Statistical analysis on raw data - averages without standard deviation or raw data cannot be checked for significance

% Raw result can go in appendices if too large, although they must be discussed if in dependencies

\section{Discussion}
% Assuming completion time indicates success
% Assuming the courses, and the nature of physical traversal indicates success
% Discussion of handedness, and preference giving strengths and weaknesses
% Discussion of symmetry vs asymmetry and how this relates to nature
% Discussion of the prototypical animal which envolved to live in a human environment, right angles etc

% 1: Discuss implications of results

% 2: Discuss robustness of experiment and therefore findings

% No results/evidence here
% Avoid speculating, make deductions justified by the data
% Conclude discussion of results

\section{Conclusion}
% Summarise work
% Avoid details
% Avoid more analysis
% Summarise the entire report, addressing the main hypothesis
% Avoid declaring with certainty stuff that is only *supported*
% Talk about the general hypothesis for robots in general, not your particular robot

%The conclusion is just one paragraph. After possible digressions in the discussion, you should come

%back to state exactly what you tried to do (brief summary of the introduction), what the outcome was

\appendix
\end{document}