\documentclass[a4paper,12pt]{article}

\usepackage{natbib}
\usepackage{times}
\usepackage{graphicx,epsfig}
\usepackage[leftcaption]{sidecap}
\usepackage{subfigure} % figures can have sub chunks
\usepackage{geometry} % this maxes page usage, making the below unnecessary
\usepackage{hhline}

\textwidth = 6.75in
\oddsidemargin = -0.25in
\textheight = 10in
\topmargin = -0.5in
\usepackage{fancyhdr}
\pagestyle{fancy}
\lhead{{\it Alex Remedios }}
\chead{Social Behaviour in Games}
\rhead{Coursework 3}
\lfoot{}
\cfoot{\thepage}
\rfoot{}


\newcommand{\goodgap}{%
 \hspace{\subfigtopskip}%
 \hspace{\subfigbottomskip}}
% For FSM diagram
\usepackage{tikz}
\usepackage{bm}

\title{Social Behaviour in Games}
\author{Alex Remedios (ajr36)}
\date{\today}

\begin{document}

% Set this to the language you want to use in your code listings (if any)
% \lstset{language=Java,breaklines,breakatwhitespace,basicstyle=\small}
\maketitle

\section{Introduction}
% Introduce problem
The task of game AI is multifaceted. Originating as a search constrained by hardware capability, proficiency is no longer the only benchmark of non-human player quality, with believability and the ability to make the game more fun also being required. This task is interpreted as the optimal division of responsibility between teammates in order to achieve a common goal; this may mean carrying out complementary tasks, or all adopting a single effective approach (although it is hypothesised the former is better for bots, as it is most real games for humans). Behaviour oriented design is used to create Unreal 2004 bots which will play capture the flag.

\section{Approach}
% the benefits and/or costs of creating a team for capture the flag, and how you helped them cooperate.
The scenario is capture the flag with teams of three. Achieving the primary objective of returning the enemy's flag to your base (whilst yours is safely there) can be attempted with `brute force'. Rudimentary behaviours involves setting all bots to attack the flag and return it, however there are auxiliary objectives which may make this easier. A rocket launcher is located on the map --- a weapon which can give the possessing team the upper hand. Our behaviours and agents must handle these conflicting goals. Furthermore this map has obstacles which can only be overcome by jumping over them; another behaviour requiring intelligence.

My approach is summarised with the following behaviours, from highest to lowest priority:
\begin{enumerate}
\item Stop shooting
\item Shoot
\item Jump
\item Find rocket launcher (Note: I didn't have time to implement this)
\item Return flag
\item Capture Flag
\item Defend flag
\item Pursue flag carrier
\item Inch
\end{enumerate}

% Avoid discussing results here, keep the methodology at the forefront

Whilst no two behaviours have the same priority, not all are mutually exclusive. If I had time, I would split the conflicting tasks of capturing/returning the flag, and defending/pursuing the flag.
\begin{itemize}
\item \textbf{Agent 1: Attacker} Engage in combat if not carrying flag, Capture flag, return flag, pursue flag carrier
\item \textbf{Agent 2: Defender} Find rocket launcher if flag is safe, Defend flag, Pursue flag carrier
\item \textbf{Agent 3: Midfielder} Find rocket launcher, pursue flag carrier, return flag
\end{itemize}

\section{Results}
% An observation should be a point about what worked or what didn?t work with respect to your task. They can be about game AI, cognitive systems more generally, cooperation more specifically. They can be informed critiques of the software tools. An observation can be of more than one sentence, so you might want to start with a hypothesis, and then describe evidence that lead you to believe it, or an experiment about how you would test it. In general, you will do better if you draw conclusions that might be applicable to more than only your own bot. Don?t feel obliged to recount exactly how your bot did in the competition?I will get that information from the tutors. The observations will be marked on a three-point scale:
% 0 Missing or redundant.
% 1 Conspicuously inaccurate or not entirely coherent, but not entirely wrong.
% 2 A good solid observation.
% 3 An exceptionally insightful point.
% Don't speculate, factual descriptions only
The three bots running the monolithic plan detailed above have the ability to jump out of their spawn point, seek and return the flag, and kill enemies which are encountered along the way.

\section{Discussions and Conclusion}

% 1: Discuss implications of results
% 2: Discuss robustness of experiment and therefore findings

% No results/evidence here
% Avoid speculating, make deductions justified by the data
% Conclude discussion of results
% Summarise work
% Avoid details
% Avoid more analysis
% Summarise the entire report, addressing the main hypothesis
% Avoid declaring with certainty stuff that is only *supported*
% Talk about the general hypothesis for robots in general, not your particular robot

%The conclusion is just one paragraph. After possible digressions in the discussion, you should come

%back to state exactly what you tried to do (brief summary of the introduction), what the outcome was
One of the main dilemmas during implementation was symmetric vs asymmetric teamwork. Team sports  regularly divide players into positions in order to firstly split responsibility optimally, and secondly use players in positions that benefit from their particular strengths. My solution uses only one plan because all bots have the same attributes and the the task affords a mob-like approach.

Capture the flag has two goals on the surface --- defend your flag, capture their flag. However, you can't complete a capture without defending your own. This causes the game to boil down to a sort of tug-of-war, where a concentrated effort (caused by a single unified plan) can lead to reasonable (although not finely tuned) success.

If I had time to spare programming this, using the rocket launcher would add an interesting element. This would give one bot much more killing power, which I would use to defend the flag. When killed, the weapon would land safely in the base for another friendly bot to acquire. A low effort and high reward to winning a game of CTF is ``parking the bus'' when you are ahead, by leaving all bots at the base and waiting for the timer to run out.

\appendix
\end{document}